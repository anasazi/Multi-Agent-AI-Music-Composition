\section{System Design}

talk about general blackboard system structure

A Blackboard System is a way for distinct agents specializing in subproblems to communicate partial solutions and cooperate to find a complete solution.
There are three components of a Blackboard System: the Blackboard, the Control, and the Knowledge Sources.
Knowledge Sources are the specialized agents. They examine the Blackboard as input and write their partial solutions on the Blackboard.
The Blackboard is a data structure shared by all the Knowledge Sources and presided over by the Control. It is essentially an open read and write area.
The Control decides which Knowledge Sources run and when the problem has been solved.

insert a diagram of a blackboard system here

\subsection{Blackboard}

talk about the specific blackboard used

For this music composition system, the Blackboard contains two major items: the given musical line, the \emph{cantus firmus}, and the musical line in the progress of composition, the \emph{counterpoint}.
The \emph{cantus firmus} is a read only list of notes along with a note-at-time lookup table for fast random access.
The \emph{counterpoint} is also a list of notes, but Knowledge Sources are allowed to extend or modify it.
Both provide a context-aware location reference (known as a Zipper) that makes it easier for Knowledge Sources and the Control to communicate about particular areas of a composition.
The Blackboard also contains other minor, but important, information such as the key or mode of the composition. This is mainly for ease of reference.

insert a diagram of the blackboard structure?

\subsection{Knowledge Sources}


talk about the general structure for agents
\subsection{Control}

talk about the control structure
