\section{Analysis of Generated Music}

\subsection{Variety} 
% no repeats, detectable difference in both audio and theory 
While deterministic AIs are useful in some contexts, variety is preferred in music composition.
Even though the first species implementation uses a non-deterministic generator, it does not follow that the system as a whole is non-deterministic.
If the rules are constraining enough, only a single result could be acceptable.
Fortunately, this does not turn out to be the case.
In testing with a single \emph{cantus firmus}, the system has never repeated a composition and has generated 32 unique compositions.
More importantly, these compositions are noticeably different when played, even to non-experts.
While entirely possible for the system to generate duplicates, it appears to be a rare event.

\subsection{Quality} 
% always finds something without violations eventually. more regular than human composed music. 
The 32 compositions discussed in the previously, in addition to being unique, made no rule violations.
That is, in every test, the system found a composition that was ``perfect'' from the perspective of first species counterpoint.
Compared to the human composition using the same \emph{cantus firmus}, several of the AI compositions sound more regular or repetitive.
Overall, the AI compositions are certainly recognizable as music and comparable with human composed equivalents, but a listener can usually differentiate between the two.

\subsection{Emergent Complexity} 
% chord progressions 
While some of the counterpoint rules deal with whole-composition properties, 
such as the ratio of skips and steps, all rules regulating structure work on a local area.
There is no conception of a whole-composition structure such as progression or cadence.
However, a noticeable property of the AI compositions is the common presence of recognizable chord progressions.
These progressions give the AI compositions a sense of structure and contribute significantly to the quality.
The common chord progressions, such as IV-V-I, appear often throughout the AI compositions.
More complex progressions, such as V-vi-IV-I, appear as well.
The presence of these progressions implies that they arise naturally from the combination of rules in first species counterpoint.

\subsection{Composition Speed} 
% fast 
Runtimes for 30 compositions of a 24 second counterpoint were recorded on a single core virtual machine.
The average runtime was 137.95 seconds with a standard deviation of 154.53 seconds.
The fastest run took 8.26 seconds and the longest took 742.99 seconds.
Five of the 30 runs took less time to compose than the composition takes to play, which requires an talented improvisationist for humans.
The longest run, 742.99 seconds or 12 minutes 23 seconds, is about how long an experienced music theory student may take to compose an equivalent piece.
The variance in runtime is primarily due to the random nature of the generator and the potentially large collection of poor generations the system will run through.

The structure of system creates many opportunities to increase performance by exploiting parallelism, although none have been implemented.
The first opportunity is the execution of testers, which all share an input, but operate independently. A parallel map/foreach would work well here.
The second opportunity is the process of finding the best blackboard, which is a scan across the (potentially very large) list to find the maximum. A parallel fold/reduce would offer significant speed up.
Those two improvements take care of the most computationally intensive parts of the current system and do not require any agents to change their implementation.
Additionally, the system could attempt to find better counterpoints more quickly by generating several new counterpoints at a time or by operating on the top several counterpoints instead of only the best.
It is common for one counterpoint to be the best for a chain of several generations, so generating several new counterpoints eliminates the best blackboard search overhead in the common case.
Operating on the top several counterpoints may help the system avoid wasting time working on a dead end.
For both changes, the parallel operations are mostly independent, with the exception of every thread needs to modify the shared blackboard metadata map.
