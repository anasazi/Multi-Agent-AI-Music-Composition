\section{Implementation Details}

% discuss what counterpoint rules are like: hard/soft, species/general, complexity, particular implementations
\subsection{Counterpoint as a Ruleset}
Counterpoint originated in the Baroque and Rennaisance as a teaching tool, so it is very structured and thoroughly studied.
The basic task in Counterpoint is to write an accompanying voice, called the counterpoint, to a provided fixed voice, called the \emph{cantus firmus}.
The form of the counterpoint has several varieties, known as species. 
For example, first species uses whole notes simultaneous with the notes in the \emph{cantus firmus}, while second pecies uses a pair of half notes.
Each species has a set of rules that define valid solution counterpoints.
These rules deal almost entirely with either intervals between the counterpoint and the \emph{cantus firmus} (a vertical interval) or intervals between two notes in the counterpoint (a horizontal interval).

Counterpoint rules come in hard and soft varieties. 
Hard rules should never be violated and usually forbid severe dissonances, excessive repetition, and degenerate counterpoints that sound too similar to the \emph{cantus firmus}.
Soft rules are style guidelines and should be followed as much as possible, but is is okay to violate them if necessary to progress or just for occasional variety.
Exmaples of soft rules are preferences like avoiding wide separation between voices, avoiding skipping in both voices simultaneously, and using stepwise motion more often that skipping motions.

% detailed look at what the Control does
\subsection{Control Strategy}

% details of how to implement blackboards efficiently
\subsection{Ordering Blackboards Efficiently}

% discuss addition of time parameter and turning agents into inductive rules
\subsection{Using a Time Parameter for Efficient Agents}

% details of how testers work
\subsection{Running Testers}

% details of how generators work
\subsection{Running Generators}
