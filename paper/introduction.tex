\section{Introduction}
This work uses multi-agent systems in a different way. Basic agents may embody a particular rule of the musical style using a particular implementation technique.
These agents interact and form a musical composition as their emergent behavior.
This approach utilizes multiple compositional techniques to overcome the weaknesses of any particular technique.
It also more easily adapted to other musical styles by adding or removing agents (i.e. different rule sets) and changing the relative importance of agents (i.e. strict rules versus guidelines).
This system combines the existing compositional techniques in order to generate higher quality music while being structured in an adaptable and extensible fashion.

Multi-agent systems may provide a way around the weaknesses of single technique systems.
Independent agents using different techniques can work together, exploiting each other's strengths and covering each other's weaknesses, to create a better composition than they would working alone.
Existing work with multi-agent computer music targets musical performance (cite performance MA system paper) or bridges the gap between performance and composition (cite real-time MA paper).
The later work is only partially computer generated as a human composer can control high level behavior during composition.
This system uses a single controlling agent (a conductor) and several worker agents (performers).
A worker agent is still using a single method of composing music, while the controller agent shifts emphasis around to highlight desirable sequences as worker agents generate them. 
Instead of overcoming the weakness of a composition method, poor output is filtered away.
This improves output by raising the quality of the worst piece instead of improving the quality of the best piece.
