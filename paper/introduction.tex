\section{Introduction}
%TODO rewrite intro to include related work references and the changes michael suggested

%Algorithmic music composition has been a tool of composers for centuries.
%Some basic examples include using dice rolls to organize small musical phrases (cite Mozart "Musical Dice" [ACM paper]) or using a mapping from syllables to notes (cite Guido d'Azzero [ACM paper]).
%The advent of computers, specifically their great data processing and storage power, enabled both automation of algorithmic compositional techniques and increased complexity and sophistication of these techniques.

%Much of the initial use of computers for composition revolved around the Serialism musical paradigm, which uses sequences (typically numerical) to organize musical elements. (cite Serialism description, ACM paper)
%Early systems combined small musical elements like a larger scale version of the "Musical Dice" idea. (cite)
%Other systems interpreted various input sources, including random noise, number sequences, and datasets, as music using some sort of mapping algorithm, again extending pre-computer ideas. (cite)
%Computers mostly enabled larger scale versions of what was done before.

%Artificial Intelligence (AI) opened up the idea that computers could reason about more complex musical styles.
%Some systems learned a style from an already existing body of works (cite Cope's EMI and others).
%Others attempted to encode the rules of a formal model of a musical style and then reason using these rules (cite).
%Several non-AI techniques, particularly cellular automata and formal grammars, were also used. (cite)
%These systems all focused on a single technique to generate music.
%Each technique tended to excel in a narrow aspect of composition, while lacking in other key areas.
%For example, artificial neural networks are excellent for generating music in a learned style, but do not generate variations on phrases well. (cite)
%Formal grammars can prevent the construction of some undesirable musical properties, but can be repetitive or have problems with unaccounted for musical properties. (cite)
%Some of these systems also have problems with more subjective musical properties like tension or color and thus can sound artificial or fake.

%Multi-agent systems may provide a way around the weaknesses of single technique systems.
%Independent agents using different techniques can work together, exploiting each other's strengths and covering each other's weaknesses, to create a better composition than they would working alone.
%Existing work with multi-agent computer music targets musical performance (cite performance MA system paper) or bridges the gap between performance and composition (cite real-time MA paper).
%The later work is only partially computer generated as a human composer can control high level behavior during composition.
%This system uses a single controlling agent (a conductor) and several worker agents (performers).
%A worker agent is still using a single method of composing music, while the controller agent shifts emphasis around to highlight desirable sequences as worker agents generate them. 
%Instead of overcoming the weakness of a composition method, poor output is filtered away.
%This improves output by raising the quality of the worst piece instead of improving the quality of the best piece.

%This work uses multi-agent systems in a different way. Basic agents may embody a particular rule of the musical style using a particular implementation technique.
%These agents interact and form a musical composition as their emergent behavior.
%This approach utilizes multiple compositional techniques to overcome the weaknesses of any particular technique.
%It also more easily adapted to other musical styles by adding or removing agents (i.e. different rule sets) and changing the relative importance of agents (i.e. strict rules versus guidelines).
%This system combines the existing compositional techniques in order to generate higher quality music while being structured in an adaptable and extensible fashion.

This work describe a multi-agent system that composes music by embodying individual musical style rules in independent agents.
The agents communicate and cooperate through a blackboard system to create a musical composition.
Splitting style rules into agents this way allows for using multiple AI and compositional techiques to overcome weaknesses of using any particular technique alone
	as well as allowing easy adaptation to other musical styles by adding or removing agents.
This can result in higher quality music than that generated by single agent systems.

Composition systems using a single technique tend to generate music that is weak in at least one aspect.
By allowing an individual agent to use whatever techniques are appropriate for its subproblem, the collection of agents can build on each other's strengths and cover each other's weaknesses.
Since all of the agents need to share the partial composition structure, a blackboard system is well suited for this system.
To demonstrate how this system may be structured, we implemented Baroque era counterpoint.
Counterpoint was originally designed for training students and is therefore a very structured and heavily documented musical style, which lends itself to this implementation method.
Basic counterpoint involves creating a musical line, which satisfies a significant collection of restrictions and preferences, to run in parallel with a provided line.
