\section{Representing Music}

The two fundamental elements of music from the theoretical perspective are notes, which define positions on the staff, a frequency, and a duration, 
and intervals, which define motion and distance between positions on the staff and between frequencies. Everything else is built upon this base.
This representation of music was derived some several texts on music theory. \cite{TonalHarmony, MusicTheoryPractice, MusicTheoryHandbook}

\subsection{Notes}
Notes are usual defined as ``a pitch over a duration'', which is true from a purely audio perspective.
However, this definition misses an important element of notes in music theory, namely that they represent positions on the staff as well.
Any two positions on the staff can represent the same pitch if enough accidentals are used, 
but even if two notes represent the same pitch, if they are at different staff positions then music theory considers them distinct.
Thus, we have: $$ Notes = Position \times Pitch \times Duration $$

Positions on the staff are specified by a letter (A-G) and an octave number. The origins of this system stem from how we hear different pitches, which we will not go into here.
Middle C is defined to be C4 and the octave number changes when moving from B to C. Therefore, the position directly below C4 is B3.
The octave number can be any integer, although any octave outside of 2 through 6 is very unusual.

Each position on the staff has a corresponding pitch, but not every pitch has a location on the staff. 
Adjacent locations are separated by either a whole-step or a half-step. Only the pairs BC and EF are separated by half-steps.
The locations separated by whole-steps hide a pitch between them.
These pitches are achieved by modifying the base (or natural) pitch of a location with \emph{accidentals}.
In music notation, accidentals are denoted with flats (lower one half-step each) or sharps (higher one half-step each).
Sometimes, double flats or double sharps are used.
If we treat an accidental as the number of half-steps to raise a pitch, then accidentals generalize nicely to any interger.
Flats are simply negative accidentals.

The duration of a note is denoted by a combination of two elements.
Basic durations are some number of halvings of a whole note.
1 whole note is 2 half notes is 4 quarter notes is 8 eighth notes and so on.
This can go arbitrarily deep, but it is unusual to see anything smaller than sixteenth notes.
The duration can be further modified by writting dots after the note.
Each dot extends the duration by 50\% of the existing duration.
A natural way to generalize duration is with a pair of nonnegative integers: the number of halvings of a whole note and the number of dots applied.
Given this pair, the actual duration as a fraction of a whole note is easily computed.

Thus, our representation of a note becomes $$ Notes = Letter \times OctaveNumber \times Accidental \times Halvings \times Dots $$

Each of the three aspects of a note provides a concept of ordering, difference, and motion along only that aspect.

\subsection{Intervals}
Intervals embody musical distance, which involves both distance on the staff and distance between pitches.
One oddity of music notation is that the distance on the staff is actually written as the number of locations spanned by the interval.
Thus, the interval that does not change location is a first and not a zeroth interval.
As with notes, each aspect of an interval provides us a concept of ordering, difference, and motion.
In music theory, intervals are denoted with their letter span and their quality, which is a property derived from the letter span and the half-step width.
We represent intervals with: $$ Intervals = LetterSpan \times HalfStepWidth $$

Intervals have an associative binary operator, interval composition, which is simply the interval created by gluing two intervals together.
This operator has an identity, the unison interval, which has a letter span of one and a half-step width of zero.
Therefore, intervals are a monoid.

Music theory has a concept of the inverse of an interval, which brings forth the question of whether intervals form a group.
However, the inverse of an interval is not a mathematical inverse. In fact, for most intervals it is not even an invertible function.
Intervals are either simple, not bigger than an octave, or compound.
The inverse of a simple interval is the interval that when the two are composed together form an octave.
Already, things break down as the inverse of the unison is the octave. That is, the inverse of the identity is not the identity.
Furthermore, the inverse of a compound interval is the inverse of the corresponding simple interval, which is the compound interval mod an octave.
Therefore, while for simple intervals the inverse of the inverse is the original interval, for compound intervals this is not the case.
Thus, intervals are certainly not a group under the usual music theory definitions.

